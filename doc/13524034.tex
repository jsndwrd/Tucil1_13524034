\documentclass[12pt,a4paper]{article}

\usepackage[margin=2.5cm]{geometry}
\setlength{\headheight}{14.5pt}
\addtolength{\topmargin}{-2.5pt}
\usepackage[T1]{fontenc}
\usepackage[utf8]{inputenc}
\usepackage[indonesian]{babel}
\usepackage{lmodern}
\usepackage{graphicx}
\usepackage{amsmath, amssymb, amsthm}
\usepackage{booktabs}
\usepackage{enumitem}
\usepackage{float}
\usepackage{fancyhdr}
\usepackage{hyperref}
\usepackage{siunitx}
\usepackage{array}
\usepackage{xcolor}
\usepackage{listings}
\usepackage{caption}
\usepackage{subcaption}
\usepackage{longtable}
\usepackage{indentfirst}
\usepackage{tabularx}

\setlength{\parindent}{1.25em}
\setlength{\parskip}{0pt}

\hypersetup{colorlinks=true, linkcolor=black, urlcolor=black, citecolor=black}

\lstset{
  basicstyle=\ttfamily\small,
  breaklines=true,
  breakatwhitespace=true,
  tabsize=2,
  showstringspaces=false,
  frame=single,
  numbers=left,
  numberstyle=\tiny,
  xleftmargin=2em,
  framexleftmargin=1.5em,
  columns=flexible
}

\pagestyle{fancy}
\fancyhf{}
\lhead{IF2211 - Strategi Algoritma}
\rhead{\thepage}

\newcommand{\Judul}{LAPORAN TUGAS KECIL 1\\IF2211 STRATEGI ALGORITMA}
\newcommand{\Subjudul}{Queens}
\newcommand{\Prodi}{Program Studi Teknik Informatika}
\newcommand{\Sekolah}{Sekolah Teknik Elektro dan Informatika}
\newcommand{\Institusi}{Institut Teknologi Bandung}
\newcommand{\Tahun}{2025}

\begin{document}

\begin{titlepage}
    \begin{center}
        
    {\Huge \textbf{Laporan Tugas Kecil 1}}\\[0.5cm]
    {\Large \textsc{IF2211 Strategi Algoritma}}\\[0.2cm]
    {\large \textsc{Penyelesaian Permainan Queens}}\\[0.2cm]
    {\large \textsc{Semester II Tahun 2025/2026}}\\[2cm]
          
    \vfill    

    {\large {Disusun oleh:}}\\[0.2cm]
    {\large
    \textbf{Jason Edward Salim - 13524034} \\[2.2cm]
    }

    {\large \textsc{Laboratorium Ilmu dan Rekayasa Komputasi}}\\[0.2cm]
    {\large \textsc{Program Studi Teknik Informatika}}\\[0.2cm]
    {\large \textsc{Sekolah Teknik Elektro dan Informatika}}\\[0.2cm]
    {\large \textsc{Institut Teknologi Bandung}}\\
    
    \end{center}
\end{titlepage}

\tableofcontents
\newpage

\section{Penerapan Algoritma Brute Force}
\subsection{Pendahuluan}
Algoritma \textit{Brute force} merupakan metode penyelesaian permasalahan secara \textit{straightforward}. Pada program penyelesaian permainan Queens ini, algoritma \textit{Brute force} digunakan secara murni dengan \textit{exhaustive search}.

\subsection{Alur Berpikir}
Queens memiliki area bujur sangkar berukuran dinamis NxN dimana kita menempatkan ratu pada tiap warna dimana setiap ratu yang ditempatkan dilarang berada pada baris dan kolom yang sama, serta dilarang berada di 8 sel di sekitar ratu yang lain. \\
Untuk mendapatkan solusi dari Queens, \textit{Brute force} murni diterapkan dengan menggunakan \textit{exhaustive search} pada tiap baris secara terus-menerus hingga ditemukan solusi yang tepat. Pada setiap baris, penempatan ratu akan dilakukan pada tiap kolom hingga baris terakhir dan dilakukan pengecekan apakah penempatan ratu sekarang sesuai dengan aturan yang ada atau tidak. Jika belum sesuai, maka penempatan berikutnya akan dilakukan hingga ditemukan solusi yang sesuai atau tidak ditemukan solusi setelah seluruh penempatan dicoba. Solusi ini akan memiliki kompleksitas waktu sekitar $O(N^N)$ karena pada tiap N baris terdapat N kolom yang dapat dicoba penempatannya.

\subsection{Implementasi Penyelesaian Brute Force}
Pada penyelesaian ini, algoritma akan terbagi menjadi:
\begin{enumerate}
    \item Fungsi utama yang akan melakukan \textit{exhaustive search} untuk penempatan ratu pada tiap baris dan kolom.
    \item Fungsi validasi penempatan yang akan mengecek apakah penempatan ratu pada suatu posisi sudah sesuai dengan aturan yang ada.
    \item Fungsi yang akan menyimpan dan mengembalikan solusi yang didapatkan.
\end{enumerate}

\subsubsection{Exhaustive Search}
Untuk setiap baris, kita akan menempatkan ratu pada setiap kolom dan melakukan validasi setelah tiap baris memiliki satu ratu. Langkah ini dapat diimplementasikan sebagai berikut.
\begin{enumerate}
    \item Iterasi tiap baris dengan kombinasi tiap kolom.
    
    Contoh: {(0, 0) hingga (0, N-1)}, {(1, 0) hingga (1, N-1)}, dan seterusnya.
    
\begin{lstlisting}[language=go]
func TryPosition(area *TArea) (queensLocation []TPosition) {
    n := area.n
    cols := make([]int, n)
    for {
        temp := make([]TPosition, n)
        for row := 0; row < n; row++ {
            temp[row] = TPosition{row, cols[row]}
        }

        if CheckPosition(*area, temp) {
            area.queensLocation = temp
            return temp
        }

        var i int
        for i = n - 1; i >= 0; i-- {
            cols[i]++
            if cols[i] < n {
                break
            }
            cols[i] = 0
        }
        if i < 0 {
            break
        }
    }

    return nil
}
\end{lstlisting}

    \item Setelah menempatkan ratu di tiap baris, validasi apakah penempatan sudah memenuhi aturan. Jika belum, iterasi ke penempatan selanjutnya.
    \item Jika solusi ditemukan, solusi akan dikembalikan dan jika tidak maka akhiri iterasi. 
    
    Pada kode di atas, fungsi \textit{CheckPosition} memvalidasi penempatan dan jika valid maka solusi akan tersimpan pada \textit{area.queensLocation} dan dikembalikan. Sebaliknya, kode akan berlanjut jika \textit{CheckPosition} mengembalikan false.

\end{enumerate}

\subsubsection{Validasi Penempatan}
Validasi yang harus dilakukan pada setiap masukan dan penempatan ratu antara lain:
\begin{enumerate}
    \item Cek apakah area yang dimasukkan memiliki ukuran NxN dengan N > 0.
    \item Cek apakah area NxN memiliki N representasi huruf.
    \item Cek apakah tiap representasi huruf selalu terhubung dan tidak terputus.
    \item Cek apakah penempatan ratu sesuai dengan aturan permainan: tidak berada pada baris dan kolom yang sama serta tiap ratu tidak terletak di 8 sel sekitar ratu lainnya.
\end{enumerate}

Langkah di atas dapat dilihat implementasinya pada folder \textbf{\textit{src/solution}} di \textit{repository} GitHub yang dapat diakses pada lampiran.

\section{Implementasi GUI}


\section{Kasus Uji}




\section*{Lampiran}
\addcontentsline{toc}{section}{Lampiran}

\begin{enumerate}
    \item Kode program dapat diakses pada: \url{https://github.com/jsndwrd/Tucil1_13524034}.
\end{enumerate}
\begin{table}[H]
\centering
\begin{tabularx}{\textwidth}{|c|X|c|c|}
\hline
\textbf{No} & \centerline{\textbf{Poin}} & \textbf{Ya} & \textbf{Tidak} \\
\hline
1 & Program berhasil dikompilasi tanpa kesalahan & \checkmark & \empty \\
\hline
2 & Program berhasil dijalankan & \checkmark & \empty \\
\hline
3 & Solusi yang diberikan program benar dan mematuhi aturan permainan & \checkmark & \empty \\
\hline
4 & Program dapat membaca masukan berkas .txt serta menyimpan solusi dalam berkas .txt &  & \empty \\
\hline
5 & Program memiliki Graphical User Interface (GUI) &  & \empty \\
\hline
6 & Program dapat menyimpan solusi dalam bentuk file gambar &  & \empty \\
\hline
\end{tabularx}
\end{table}

Tugas ini disusun sepenuhnya tanpa bantuan kecerdasan buatan \textit{(Generative AI)}, melainkan hasil pemikiran dan analisis mandiri.

\begin{flushright}
    \begin{figure}[h!]
        \raggedleft
        \includegraphics[width=0.3\textwidth]{signature.png}
    \end{figure}
    \textbf{Jason Edward Salim} \\
\end{flushright}

\end{document}